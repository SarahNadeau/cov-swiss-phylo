\documentclass[9pt,twoside,lineno]{pnas-new}
% Use the lineno option to display guide line numbers if required.

\templatetype{pnassupportinginfo}

\title{Your main manuscript title}
\author{Author1, Author2 and Author3 (complete author list)}
\correspondingauthor{Corresponding Author name.\\E-mail: author.two@email.com}

% Main analysis
\newcommand\mindate{25. Feb. 2020}
\newcommand\maxdate{31. Dec. 2020}
\newcommand\maxsamplingpercent{1}
\newcommand\gensimscalefactor{0.5}
\newcommand\travelcontextscalefactor{1}
\newcommand\outgroupisl{EPI\_ISL\_406798}
\newcommand\nswissseqs{X}
\newcommand\nsimseqs{Y}
\newcommand\ntravelseqs{Z}
\GISAIDpulldate{DD. Month 2021}

\newcommand\nchainsmin{296}
\newcommand\nchainsmax{1058}
\newcommand\minlargestchainsper{48}
\newcommand\maxlargestchainsper{27}
\newcommand\nspanningchainsmin{5}
\newcommand\nspanningchainsmax{1}

\newcommand\ncinongruentexposurechainsmax{7} 
\newcommand\nexposurechainsmax{106}
\newcommand\ncinongruentexposurechainsmin{14} 
\newcommand\nexposurechainsmin{64}
\newcommand\rankexpsamplemin{2}
\newcommand\rankexpsamplemax{1}

% Validation analysis
\newcommand\mindateval{25. Feb. 2020}
\newcommand\maxdateval{31. Dec. 2020}
\newcommand\maxsamplingpercentval{0.5}
\newcommand\gensimscalefactorval{1}
\newcommand\travelcontextscalefactorval{0.5}
\newcommand\travelcontextweightsval{1:1:0} % exposures, tourists, commuters
\newcommand\outgroupislval{EPI\_ISL\_406798}
\newcommand\nswissseqsval{X}
\newcommand\nsimseqsval{Y}
\newcommand\ntravelseqsval{Z}

\begin{document}

%% Comment out or remove this line before generating final copy for submission; this will also remove the warning re: "Consecutive odd pages found".
\instructionspage  

\maketitle

%% Adds the main heading for the SI text. Comment out this line if you do not have any supporting information text.
\SItext


% \subsection*{Subhead}
% Type or paste text here. This should be additional explanatory text such as an extended technical description of results, full details of mathematical models, etc.   
\section{Sensitivity analyses}
We investigated the sensitivity of transmission chain summary statistics to different ratios of focal Swiss sequences to genetically similar foreign context sequences. Figure \ref{fig:sensitivity_context_set_size} shows that as the size of the context set is increased from a 1:1 ratio to a 3:1 ratio, the number of transmission chains estimated increases and the mean size of transmission chains decreases. However, these differences are smaller than the differences within each analysis, depending on whether the smallest or largest plausible transmission chains are assumed. Therefore, we chose the 2:1 ratio as a balance of speed (smaller dataset = faster tree search convergence) and information content (larger dataset = more precise estimates of transmission chain import dates) and rely on the different transmission chain assumptions within this analysis to incorporate most of the uncertainty in transmission chain definition.

\begin{figure*}[tbhp]
\centering
\includegraphics[width = 11.4cm]{figures/fig_SX_sensitivity_context_set_size.png}
\caption{Sensitivity of transmission chain summary statistics to different ratios of focal Swiss sequences to genetically similar foreign context sequences (1:1, 1:2, and 1:3).}  
\label{fig:sensitivity_context_set_size}
\end{figure*}



We investigated the sensitivity of transmission chain summary statistics to the number of focal Swiss sequences analyzed.


\section*{Heading}
\subsection*{Subhead}
Type or paste text here. You may break this section up into subheads as needed (e.g., one section on ``Materials'' and one on ``Methods'').

\subsection*{Materials}
Add a materials subsection if you need to.

\subsection*{Methods}
Add a methods subsection if you need to.


%%% Each figure should be on its own page
\begin{figure}
\centering
\includegraphics[width=\textwidth]{example-image}
\caption{First figure}
\end{figure}

\begin{figure}
\centering
\includegraphics[width=\textwidth]{frog}
\caption{Second figure}
\end{figure}

\begin{table}\centering
\caption{This is a table}

\begin{tabular}{lrrr}
Species & CBS & CV & G3 \\
\midrule
1. Acetaldehyde & 0.0 & 0.0 & 0.0 \\
2. Vinyl alcohol & 9.1 & 9.6 & 13.5 \\
3. Hydroxyethylidene & 50.8 & 51.2 & 54.0\\
\bottomrule
\end{tabular}
\end{table}

%%% Add this line AFTER all your figures and tables
\FloatBarrier

\movie{Type legend for the movie here.}

\movie{Type legend for the other movie here. Adding longer text to show what happens, to decide on alignment and/or indentations.}

\movie{A third movie, just for kicks.}

\dataset{dataset_one.txt}{Type or paste legend here.}

\dataset{dataset_two.txt}{Type or paste legend here. Adding longer text to show what happens, to decide on alignment and/or indentations for multi-line or paragraph captions.}

\bibliography{pnas-sample}

\end{document}
